As a summary this work has made a careful and systematic study on the multifractal characteristics at the critical point and in the critical regime of the electronic state within the Anderson model of localisation using high-precision data and very large system sizes.  It has demonstrated that the best strategy to obtain a complete picture of the multifracta spectrum, which contains negative fractal dimensions and shows the contributions coming from the tails of the distribution, is to use system-size scaling with ensemble averaging in which to specifically consider a sensible range of system sizes with very large number of states.
From the multifractal analysis, we extract information about critical properties of the Anderson transition such as the following: validity of the symmetry relation in the multifractal exponents and probability distribution function (PDF) of wavefunction intensities, existence of rare events which give rise to negative fractal dimensions, critical disorder and critical exponent which describes the divergence of the correlation length near the critical point.  The multifractal analysis has also allowed us to comment and give speculations about the non-parabolicity of the multifractal spectrum, non-Gaussian nature of the PDF and possibility of termination points in the $f(\alpha)$.  Lastly, this work has proposed an alternative method that is numerically simpler to obtain the multifractal spectrum from the PDF of wavefunction intensities where the $f(\alpha)$ is interpreted to be the scale-invariant distribution at criticality.